\documentclass{article}
\usepackage[utf8]{inputenc}
\usepackage[a5paper, left = 1cm, right = 1cm, bottom = 2cm, top = 2cm]{geometry}
\usepackage[russian]{babel}
\usepackage[T2A]{fontenc}
\usepackage{tikz}
\usepackage{pgfplots}
\usepackage{listings}
\usepackage{fancyhdr}

\pagestyle{empty}
\lstset{ 
	backgroundcolor = \color{white},   
	basicstyle = \scriptsize\ttfamily, 
	frame = single,	                 
	keywordstyle = \color{blue},       
	language = Python,
	numbers = left,                    
	numbersep = 5pt,                   
	numberstyle = \tiny \ttfamily \color{black}, 
	stringstyle = \color{red},
	tabsize = 2
}
\pagestyle{fancy}
\fancyhf{}
\fancyhead[L]{\textit{Долгопрудный}}
\fancyhead[C]{\large{\textbf{Числа и картинки}}}
\fancyhead[R]{\textit{09.11.18}}
\fancyfoot[R]{\textcopyright Арслан Хабутдинов}

\begin{document}
    \par Код на питоне, который рисует красивую (или не очень картинку), и, вместе с тем, сам по себе очень даже ничего. Он демонстрирует невероятную простоту и универсательность этого  популярного языка программирования.
    \lstinputlisting[language = Python]{gen.py}
    \par А вот и красивая картинка, которую этот код рисует. Тут есть точечки и приближающая их прямая. Говорят, есть какие-то ряды Тейлора, которые позволяют приблизить точнее, но тогда на картинке уже не прямая получится.
    \center{
        \resizebox{0.5\textwidth}{!}{
            \begin{tikzpicture}
                \begin{axis}
                    \addplot+ [x = x, y = y, only marks, mark size = 1pt] table{data.dat};
                    \addplot[no marks, domain = -10:10, green, line width = 2pt]{3.0612*x + 2.6882};
                \end{axis}
            \end{tikzpicture}
        }
    }
\end{document}
