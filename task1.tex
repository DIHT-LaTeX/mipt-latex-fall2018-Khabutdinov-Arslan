\documentclass[12pt]{article}
\usepackage[russian]{babel}
\usepackage[utf8]{inputenc}

\title{Домашняя работа №1}
\author{Арслан Хабутдинов}
\date{}

\begin{document}
	\maketitle
	\begin{flushright}
	{\small{\textit{Audi multa, \\ loquere pauca}}}
	\end{flushright}
	\vspace{10pt}
	\begin{center}
	Это мой первый документ в системе компьютерной вёрстки \LaTeX.
	\end{center}
	
	\begin{center}
	\vspace{4pt} 
	{\Huge \sffamily 
	<<Ура!!!>>}
	\end{center}
	
	\hspace{14pt} А теперь формулы. \textsc{Формула}~--- краткое и точное словесное выражение, определение или же ряд математических величин, выраженный условными знаками.
	\vspace{15pt}
	
	{\hspace{14pt} \bfseries \Large Термодинамика}
	
	Уравнение Менделеева--Клапейрона~--- уравнение состояния идеального года, имеющее вид $pV = \nu RT$, где $p$~--- давление, $V$~--- объем, занимаемый газом, $T$~--- температура газа, $\nu$~--- количество вещества газа, а $R$~--- универсальная газовая постоянная.
	\vspace{15pt}
	
	{\hspace{14pt} \bfseries \Large Геометрия \hfill Планиметрия}
	
	Для \textsl{плоского} треугольника со сторонами $a$, $b$, $c$ и углом $\alpha$, лежащим против стороны $a$, справедливо соотношение:
	
	$$
	a ^ 2 = b ^ 2 - 2bc \cos{\alpha},
	$$
	
	из которого можно выразить косинус угла треугольника:
	
	$$
	\cos{\alpha} = \frac{b ^ 2 + c ^ 2 - a ^ 2}{2bc}.
	$$
	
	Пусть $p$~--- полупериметр треугольника, тогда путем несложных преобразований можно получить, что 
	
	$$
	\tg{\frac{\alpha}{2}} = \sqrt{\frac{(p - b)(p - c)}{p(p - a)}},
	$$
	
	\vspace{1cm}
	На сегодня, пожалуй, хватит\dots Удачи!
\end{document}

