\documentclass[11pt]{article}
\usepackage[utf8]{inputenc}
\usepackage[russian]{babel}
\usepackage{amsmath}
\usepackage{amsthm}
\usepackage{amsfonts}
\usepackage{amssymb}
\usepackage{textcomp}
\usepackage{marvosym}
\usepackage{graphicx}
\usepackage[
a5paper,
left = 1 cm,
right = 1 cm,
top = 1 cm,
bottom = 1 cm,
]{geometry}

\title{Домашняя работа №2}
\author{Арслан Хабутдинов}
\date{}

\newtheorem{thm}{Теорема}[section]
\newtheorem{dfn}{Определение}
\newtheorem{lem}{Лемма}
\newcounter{astron}[section]
\newcommand{\heart}{\ensuremath\heartsuit}
\pagenumbering{gobble}
\newenvironment{astro}[1]{\textbf{\stepcounter{astron}Задача \theastron. #1 \vspace{5pt}\\}}{\\\\}

\begin{document}
    \maketitle
    \section{Центральная предельная теорема}
    \begin{thm}[Линдеберга]
        Пусть $\{\xi_k\}_{k \geq 1}$~--- независимая случайная величина, $\mathsf{E}\xi_k < +\infty$, $\forall k$ обозначим $m_k = \mathsf{E}\xi_k$, $\sigma_k^2 = \mathsf{D}\xi_k > 0:S_n = \sum\limits_{i = 1}^n\xi_i$; $D_n^2 = \sum\limits_{k = 1}^n\sigma_k^2$ и $F_k(x)$ функция распределения $\xi_k$. Пусть выполняется условие Линдберга, то есть 
        $$
        \forall\varepsilon > 0~\frac{1}{\mathsf{D}_n^2} \sum\limits_{k = 1}^n \int\limits_{\{x:|x - m_k| > \varepsilon \mathsf{D}_n\}}^{} (x - m_k)^2  \,d f_k(x) \xrightarrow[n\rightarrow\infty]{}0.
        $$
        Тогда $\frac{S_n - \mathsf{E}S_n}{\sqrt{\mathsf{D}S_n}} \xrightarrow{d} \mathcal{N}(0,1), n \rightarrow \infty.$
    \end{thm}

    \section{Гауссовские случайные векторы}
    \begin{dfn}
        Случайный вектор $\vec \xi \sim (m, \Sigma)$ ~--- гаусс, если его характеристическая функция $\varphi_\xi\biggl(\vec t\biggr) = \exp{\biggl(i\biggl(\vec m, \vec t} - \frac{1}{2}\biggl(\Sigma \vec t, \vec t\biggr)\biggr)$, $\vec m \in \mathbb{R}^n$, $\Sigma$ ~--- симметричная неотрицательно определенная  матрица.
    \end{dfn}
    \begin{dfn}
        Случайный вектор $\xi \sim \mathcal{N}(0,1)$~--- гаусс, если он представляется в следующем виде: $\vec \xi = A\vec \eta + \vec b$, где  $\vec b \in \mathbb{R}^n$, $A \in \text{Mat}_{(n \times m)}$ и $\vec \eta = (\eta_1, \dots, \eta_m)$~---независимы и $n \sim \mathcal{N}(0,1)$.
    \end{dfn}
    \begin{dfn}
        Случайный вектор $\vec \xi$~--- гаусс, если  $\forall \lambda \in \mathbb{R}^n$ случайная величина,$\biggl(\vec\lambda, \vec\xi\biggr)$ имеем нормальное распределение.
    \end{dfn}
    \begin{thm}[об эквивалентности определений гаусс векторов]
        Предыдущие три определения эквивалентны.
    \end{thm}
    \section{Астрономия}
    \begin{astro}{Загадочный круг}
        Установите астрономический азимут восхода звезды $\varepsilon$ CМa ($6^h58^m38^s$, $-28^\circ58'$) при наблюдении из самой северной равноудалённой от Санкт-Петербурга ($59^\circ57'$ с.ш., $30^\circ19'$ в.д.) и Красной Поляны ($43^\circ41'$ с.ш., $40^\circ11'$ в.д.) точки земной поверхности. Атмосферой пренебрегите, Земля~--- шар.
    \end{astro}
    \begin{astro}{Бейрут}
        В какой момент по истинному солнечному времени 1 сентября Регул ($\alpha_1 = 10^h9^m$, $\delta_1 = 11^\circ53'$) и Шератан ($\alpha_2 = 11^h15^m$, $\delta_2 = 15^\circ20'$) находятся на одном альмукантарате в Бейруте ($\delta = 33^\circ53'$)?
    \end{astro}
    \begin{astro}{К Сатурну!}
        Космический корабли запустили с поверхности Земли к Сатурну по наиболее энергетически выгодной траектории. При движении по орбите корабль пролетел мимо астероида-троянца (624) Гектор.
    \noindentОпределите большую полуось и эксцентриситет полученной орбиты, скорость старта с поверхности Земли, а также угол между направлением на Солнце и на Сатурн в момент старта корабля. Орбиты планет считать круговыми. Оцените относительную скорость корабля и астероида в момент сближения.
    \end{astro}
    \begin{astro}{H II}
        Обратным эффектом Комптона (ОЭК) называют явление рассеяния фотона на ультрарелятивистском свободном электроне, при котором происходит перенос энергии от электрона к фотону.  Рассмотрите ОЭК для фотонов реликтового излучения. При какой энергии электронов в направленном пучке рассеянное излучение можно будет зарегистрировать на фотоприемнике?
    \end{astro}

    \section{Отзыв}
    \begin{itemize}
        \item[$\heart$] Мне нравится этот интенсивный курс.
        \item[$\clubsuit$] Он организован так, что порой не обойтись без подорожника.
        \item[$\bigstar$] Материал увлекательный, 5 \LaTeX ~звезд!
    \end{itemize}

\end{document}
